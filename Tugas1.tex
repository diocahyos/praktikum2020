% Options for packages loaded elsewhere
\PassOptionsToPackage{unicode}{hyperref}
\PassOptionsToPackage{hyphens}{url}
%
\documentclass[
]{article}
\usepackage{lmodern}
\usepackage{amssymb,amsmath}
\usepackage{ifxetex,ifluatex}
\ifnum 0\ifxetex 1\fi\ifluatex 1\fi=0 % if pdftex
  \usepackage[T1]{fontenc}
  \usepackage[utf8]{inputenc}
  \usepackage{textcomp} % provide euro and other symbols
\else % if luatex or xetex
  \usepackage{unicode-math}
  \defaultfontfeatures{Scale=MatchLowercase}
  \defaultfontfeatures[\rmfamily]{Ligatures=TeX,Scale=1}
\fi
% Use upquote if available, for straight quotes in verbatim environments
\IfFileExists{upquote.sty}{\usepackage{upquote}}{}
\IfFileExists{microtype.sty}{% use microtype if available
  \usepackage[]{microtype}
  \UseMicrotypeSet[protrusion]{basicmath} % disable protrusion for tt fonts
}{}
\makeatletter
\@ifundefined{KOMAClassName}{% if non-KOMA class
  \IfFileExists{parskip.sty}{%
    \usepackage{parskip}
  }{% else
    \setlength{\parindent}{0pt}
    \setlength{\parskip}{6pt plus 2pt minus 1pt}}
}{% if KOMA class
  \KOMAoptions{parskip=half}}
\makeatother
\usepackage{xcolor}
\IfFileExists{xurl.sty}{\usepackage{xurl}}{} % add URL line breaks if available
\IfFileExists{bookmark.sty}{\usepackage{bookmark}}{\usepackage{hyperref}}
\hypersetup{
  pdftitle={Tugas 2},
  pdfauthor={Dio Cahyo Saputra},
  hidelinks,
  pdfcreator={LaTeX via pandoc}}
\urlstyle{same} % disable monospaced font for URLs
\usepackage[margin=1in]{geometry}
\usepackage{color}
\usepackage{fancyvrb}
\newcommand{\VerbBar}{|}
\newcommand{\VERB}{\Verb[commandchars=\\\{\}]}
\DefineVerbatimEnvironment{Highlighting}{Verbatim}{commandchars=\\\{\}}
% Add ',fontsize=\small' for more characters per line
\usepackage{framed}
\definecolor{shadecolor}{RGB}{248,248,248}
\newenvironment{Shaded}{\begin{snugshade}}{\end{snugshade}}
\newcommand{\AlertTok}[1]{\textcolor[rgb]{0.94,0.16,0.16}{#1}}
\newcommand{\AnnotationTok}[1]{\textcolor[rgb]{0.56,0.35,0.01}{\textbf{\textit{#1}}}}
\newcommand{\AttributeTok}[1]{\textcolor[rgb]{0.77,0.63,0.00}{#1}}
\newcommand{\BaseNTok}[1]{\textcolor[rgb]{0.00,0.00,0.81}{#1}}
\newcommand{\BuiltInTok}[1]{#1}
\newcommand{\CharTok}[1]{\textcolor[rgb]{0.31,0.60,0.02}{#1}}
\newcommand{\CommentTok}[1]{\textcolor[rgb]{0.56,0.35,0.01}{\textit{#1}}}
\newcommand{\CommentVarTok}[1]{\textcolor[rgb]{0.56,0.35,0.01}{\textbf{\textit{#1}}}}
\newcommand{\ConstantTok}[1]{\textcolor[rgb]{0.00,0.00,0.00}{#1}}
\newcommand{\ControlFlowTok}[1]{\textcolor[rgb]{0.13,0.29,0.53}{\textbf{#1}}}
\newcommand{\DataTypeTok}[1]{\textcolor[rgb]{0.13,0.29,0.53}{#1}}
\newcommand{\DecValTok}[1]{\textcolor[rgb]{0.00,0.00,0.81}{#1}}
\newcommand{\DocumentationTok}[1]{\textcolor[rgb]{0.56,0.35,0.01}{\textbf{\textit{#1}}}}
\newcommand{\ErrorTok}[1]{\textcolor[rgb]{0.64,0.00,0.00}{\textbf{#1}}}
\newcommand{\ExtensionTok}[1]{#1}
\newcommand{\FloatTok}[1]{\textcolor[rgb]{0.00,0.00,0.81}{#1}}
\newcommand{\FunctionTok}[1]{\textcolor[rgb]{0.00,0.00,0.00}{#1}}
\newcommand{\ImportTok}[1]{#1}
\newcommand{\InformationTok}[1]{\textcolor[rgb]{0.56,0.35,0.01}{\textbf{\textit{#1}}}}
\newcommand{\KeywordTok}[1]{\textcolor[rgb]{0.13,0.29,0.53}{\textbf{#1}}}
\newcommand{\NormalTok}[1]{#1}
\newcommand{\OperatorTok}[1]{\textcolor[rgb]{0.81,0.36,0.00}{\textbf{#1}}}
\newcommand{\OtherTok}[1]{\textcolor[rgb]{0.56,0.35,0.01}{#1}}
\newcommand{\PreprocessorTok}[1]{\textcolor[rgb]{0.56,0.35,0.01}{\textit{#1}}}
\newcommand{\RegionMarkerTok}[1]{#1}
\newcommand{\SpecialCharTok}[1]{\textcolor[rgb]{0.00,0.00,0.00}{#1}}
\newcommand{\SpecialStringTok}[1]{\textcolor[rgb]{0.31,0.60,0.02}{#1}}
\newcommand{\StringTok}[1]{\textcolor[rgb]{0.31,0.60,0.02}{#1}}
\newcommand{\VariableTok}[1]{\textcolor[rgb]{0.00,0.00,0.00}{#1}}
\newcommand{\VerbatimStringTok}[1]{\textcolor[rgb]{0.31,0.60,0.02}{#1}}
\newcommand{\WarningTok}[1]{\textcolor[rgb]{0.56,0.35,0.01}{\textbf{\textit{#1}}}}
\usepackage{graphicx,grffile}
\makeatletter
\def\maxwidth{\ifdim\Gin@nat@width>\linewidth\linewidth\else\Gin@nat@width\fi}
\def\maxheight{\ifdim\Gin@nat@height>\textheight\textheight\else\Gin@nat@height\fi}
\makeatother
% Scale images if necessary, so that they will not overflow the page
% margins by default, and it is still possible to overwrite the defaults
% using explicit options in \includegraphics[width, height, ...]{}
\setkeys{Gin}{width=\maxwidth,height=\maxheight,keepaspectratio}
% Set default figure placement to htbp
\makeatletter
\def\fps@figure{htbp}
\makeatother
\setlength{\emergencystretch}{3em} % prevent overfull lines
\providecommand{\tightlist}{%
  \setlength{\itemsep}{0pt}\setlength{\parskip}{0pt}}
\setcounter{secnumdepth}{-\maxdimen} % remove section numbering

\title{Tugas 2}
\author{Dio Cahyo Saputra}
\date{2/17/2020}

\begin{document}
\maketitle

Import dataset ``murders'':

\begin{Shaded}
\begin{Highlighting}[]
\KeywordTok{library}\NormalTok{(dslabs)}
\KeywordTok{data}\NormalTok{(murders)}
\end{Highlighting}
\end{Shaded}

\hypertarget{soal-nomor-1}{%
\subsubsection{Soal Nomor 1}\label{soal-nomor-1}}

Penjelasan tentang fungsi str(), uji coba dengan object murders

\begin{Shaded}
\begin{Highlighting}[]
\KeywordTok{str}\NormalTok{(murders)}
\end{Highlighting}
\end{Shaded}

\begin{verbatim}
## 'data.frame':    51 obs. of  5 variables:
##  $ state     : chr  "Alabama" "Alaska" "Arizona" "Arkansas" ...
##  $ abb       : chr  "AL" "AK" "AZ" "AR" ...
##  $ region    : Factor w/ 4 levels "Northeast","South",..: 2 4 4 2 4 4 1 2 2 2 ...
##  $ population: num  4779736 710231 6392017 2915918 37253956 ...
##  $ total     : num  135 19 232 93 1257 ...
\end{verbatim}

Fungsi str() digunakan untuk menampilkan struktur internal objek R,
fungsi diagnostik, dan alternatif ringkasan. Idealnya, hanya satu baris
untuk setiap struktur `dasar' yang ditampilkan. Ini sangat cocok untuk
secara kompak menampilkan isi daftar. Idenya adalah untuk memberikan
output yang masuk akal untuk objek R. Itu memanggil args untuk objek
fungsi.

\hypertarget{soal-nomor-2}{%
\subsubsection{Soal Nomor 2}\label{soal-nomor-2}}

Sebutkan apa saja nama kolom yang digunakan pada data frame

Kolom pada data frame antara lain state, abb, region, population, dan
total data yang dapat dilihat pada fungsi str()

\begin{Shaded}
\begin{Highlighting}[]
\KeywordTok{str}\NormalTok{(murders)}
\end{Highlighting}
\end{Shaded}

\begin{verbatim}
## 'data.frame':    51 obs. of  5 variables:
##  $ state     : chr  "Alabama" "Alaska" "Arizona" "Arkansas" ...
##  $ abb       : chr  "AL" "AK" "AZ" "AR" ...
##  $ region    : Factor w/ 4 levels "Northeast","South",..: 2 4 4 2 4 4 1 2 2 2 ...
##  $ population: num  4779736 710231 6392017 2915918 37253956 ...
##  $ total     : num  135 19 232 93 1257 ...
\end{verbatim}

\hypertarget{soal-nomor-3}{%
\subsubsection{Soal Nomor 3}\label{soal-nomor-3}}

Gunakan operator aksesor (\$) untuk mengekstrak informasi singkatan
negara dan menyimpannya pada objek ``a''. Sebutkan jenis class dari
objek tersebut

Disini kolom singkatan negara adalah ``abb'' jadi menggunakan

\begin{Shaded}
\begin{Highlighting}[]
\NormalTok{a =}\StringTok{ }\NormalTok{murders}\OperatorTok{$}\NormalTok{abb}
\NormalTok{a}
\end{Highlighting}
\end{Shaded}

\begin{verbatim}
##  [1] "AL" "AK" "AZ" "AR" "CA" "CO" "CT" "DE" "DC" "FL" "GA" "HI" "ID" "IL" "IN"
## [16] "IA" "KS" "KY" "LA" "ME" "MD" "MA" "MI" "MN" "MS" "MO" "MT" "NE" "NV" "NH"
## [31] "NJ" "NM" "NY" "NC" "ND" "OH" "OK" "OR" "PA" "RI" "SC" "SD" "TN" "TX" "UT"
## [46] "VT" "VA" "WA" "WV" "WI" "WY"
\end{verbatim}

Lalu memasukkan objek ``a'' pada fungsi class()

\begin{Shaded}
\begin{Highlighting}[]
\KeywordTok{class}\NormalTok{(a)}
\end{Highlighting}
\end{Shaded}

\begin{verbatim}
## [1] "character"
\end{verbatim}

Jadi pada operator
``\(" itu digunakan sebagai pemisah dan untuk memanggil murders pada kolom abb "murders\)abb''
dan dimasukkan ke variable ``a''. Lalu pada fungsi class() digunakan
untuk mengetahui jenis class tersebut.

\hypertarget{soal-nomor-4}{%
\subsubsection{Soal Nomor 4}\label{soal-nomor-4}}

Gunakan tanda kurung siku untuk mengekstrak singkatan negara dan
menyimpannya pada objek ``b''. Tentukan apakah variabel ``a'' dan ``b''
bernilai sama?

\begin{Shaded}
\begin{Highlighting}[]
\NormalTok{b<-murders[,}\DecValTok{2}\NormalTok{]}
\KeywordTok{all}\NormalTok{(a}\OperatorTok{==}\NormalTok{b)}
\end{Highlighting}
\end{Shaded}

\begin{verbatim}
## [1] TRUE
\end{verbatim}

Murders diekstrak menggunakan murders{[},2{]} maksudnya didalam memilih
semua baris di kolom 2 yaitu singkatan negara atau abb, kemudian fungsi
all(a==b) artinya nilai yang berada di variable a dibangingkan dengan b
dan hasilnya sama.

\hypertarget{soal-nomor-5}{%
\subsubsection{Soal Nomor 5}\label{soal-nomor-5}}

Variabel region memiliki tipe data: factor. Dengan satu baris kode,
gunakan fungsi level dan length untuk menentukan jumlah region yang
dimiliki dataset.

\begin{Shaded}
\begin{Highlighting}[]
\NormalTok{x =}\StringTok{ }\KeywordTok{levels}\NormalTok{(murders}\OperatorTok{$}\NormalTok{region)}
\NormalTok{x}
\end{Highlighting}
\end{Shaded}

\begin{verbatim}
## [1] "Northeast"     "South"         "North Central" "West"
\end{verbatim}

\begin{Shaded}
\begin{Highlighting}[]
\KeywordTok{length}\NormalTok{(x)}
\end{Highlighting}
\end{Shaded}

\begin{verbatim}
## [1] 4
\end{verbatim}

Fungsi levels() menyediakan akses ke atribut level variabel. Bentuk
pertama mengembalikan nilai level argumennya dan yang kedua menetapkan
atribut. Lalu fungsi length() digunakan untuk mendapatkan jumlah data.

\hypertarget{soal-nomor-6}{%
\subsubsection{Soal Nomor 6}\label{soal-nomor-6}}

Fungsi table dapat digunakan untuk ekstraksi data pada tipe vektor dan
menampilkan frekuensi dari setiap elemen. Dengan menerapkan fungsi
tersebut, dapat diketahui jumlah state pada tiap region. Gunakan fungsi
table dalam satu baris kode untuk menampilkan tabel baru yang berisi
jumlah state pada tiap region

\begin{Shaded}
\begin{Highlighting}[]
\KeywordTok{table}\NormalTok{(murders}\OperatorTok{$}\NormalTok{region)}
\end{Highlighting}
\end{Shaded}

\begin{verbatim}
## 
##     Northeast         South North Central          West 
##             9            17            12            13
\end{verbatim}

Fungsi tabel() menggunakan lintas-faktor mengklasifikasikan untuk
membangun tabel kontingensi dari jumlah pada setiap kombinasi dari
tingkat faktor.

\end{document}
